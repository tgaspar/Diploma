
Na sliki \ref{fig:twocoordinates} sta dva koordinatna sistema $\textbf{O}_0$ in $\textbf{O}_1$. Želja je opisati točko v prostoru 
\begin{equation} \label{eq:p-point}
	\textbf{P} = 
	\begin{bmatrix}
		P_x \\
		P_y \\
		P_z \\
	\end{bmatrix}.
\end{equation}

Vektor translacije
\begin{equation} \label{eq:o0-to-01}
\textbf{o}_1^0 = 
\begin{bmatrix}
{\textbf{o}_1^0}_x \\
{\textbf{o}_1^0}_y \\
{\textbf{o}_1^0}_z \\
\end{bmatrix} 
\end{equation} 
in matrika rotacije

\begin{equation} \label{eq:r01}
\textbf{R}_1^0 = 
\begin{bmatrix}
c_\varphi c_\vartheta c_\psi - s_\varphi s_\psi & - c_\varphi c_\vartheta s_\psi - s_\varphi s_\psi && c_\varphi s_\vartheta \\
s_\varphi c_\vartheta c_\psi - c_\varphi s_\psi & - s_\varphi c_\vartheta s\psi + c_\varphi c_\psi && s_\varphi s_\vartheta \\
- s_\vartheta c_\psi & s_\vartheta s_\psi && c_\vartheta \\
\end{bmatrix}
\end{equation} 

opišeta relacijo med $\textbf{O}_0$ in $\textbf{O}_1$. V \ref{eq:r01} so uporabljene okrajšave za kosinusne ($c$) in sinusne ($s$) funkcije Eulerjevih kotov.

S poznavanjem $\textbf{o}_1^0$ in $\textbf{R}_1^0$ se lahko točko $\textbf{P}$, ki je v začetku izražena v koordinatnemu sistemu $\textbf{O}_0$ izrazi v koordinatnemu sistemu $\textbf{O}_1$ s sledečo transoformacijo:

\begin{equation} \label{eq:basic-transofrm}
\textbf{p}^0 = \textbf{o}_1^0 + \textbf{R}_1^0 \cdot \textbf{p}^1.
\end{equation}

Z upoštevanjem ortogonalnosti matrike  $\textbf{R}_1^{0T}$, za katero velja

\begin{equation} \label{eq:r-matrix}
{\textbf{R}_1^{0}}^{-1} = {\textbf{R}_1^{0}}^{T}
\end{equation}

in množenjem obeh strani enačbe z ${\textbf{R}_1^{0}}^{-1}$ se lahko izpostavi $\textbf{p}^1$ na sledeč način:

\begin{equation} \label{eq:basic-inverse-transofrm}
\textbf{p}^1 = - {\textbf{R}_1^{0}}^{T} \cdot \textbf{o}_1^0 + {\textbf{R}_1^{0}}^{T} \cdot \textbf{p}^0.
\end{equation}